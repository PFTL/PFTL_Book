\chapter{Python For The Lab {DAQ} Device Manual}\label{python-for-the-lab-daq-devicemanual}
We believe that learning how to program software for a scientific
laboratory can be achieved only through real-world examples. That is why
the course was conceived around a small device that we are calling a
General {DAQ} Device. In these pages, we document the behavior of the
device, in a similar way to what you would normally find in the manual
of any instrument in your own lab.

\section{Capabilities}\label{capabilities}
The \textbf{General {DAQ} Device} is a multi-purpose acquisition card
that can handle digital and analog inputs and outputs. It runs on an
{ARM} 32-bit microprocessor and drives its power from a {USB}
connection. The device can handle one task per turn, but the outputs are
persistent. This means that if you set the output of a particular port,
it will remain constant until a command for changing it is issued.

Normal Analog-to-Digital conversion times are in the order of 10
microseconds and are done with a resolution of 10 bits in the range
0-3.3V. Digital to Analog conversions are done with a resolution of 12
bits in the range 0-3.3V.

The {DAQ} possesses 2 Analog Output channels and 10 Analog Input
channels. Each can be addressed independently, however, a degree of
crosstalk can be observed, especially between neighboring ports.
Sensitive applications would, therefore, need to use non-consecutive
ports in order to mitigate this effect.

\section{Communication with a computer}\label{communication-with-acomputer}
The {DAQ} is able to communicate with the computer through a {USB}
connection. However the device has an onboard chip that converts the
communication into serial, therefore it will appear listed as any other
serial device.

The baud rate has to be set to 115200, and every command has to finish
with the newline {ASCII} character. The messages generated by the device
are also terminated by a newline character.

\section{List of Commands Available}\label{list-of-commandsavailable}
\textbf{{IDN}}: Identifies the device; returns a string with information
regarding the serial number and version of the firmware. \emph{Returns}:
String with information

\textbf{{OUT}:}: Command for setting the output of an analog channel. It
takes as arguments the channel \textbf{{CH}:\{\}} and the value,
\textbf{\{\textless{}4\}}. The value has to be in the range 0-4095,
while the channel has to be either 0 or 1.

\emph{Example}: {OUT}:{CH0}:1024

\textbf{{IN}}: Command for reading an analog input channel. It takes one
argument, \textbf{{CH}:\{\}}, between 0 and 9. \emph{Returns}: integer
in the range 0-1024

\emph{Example}: {IN}:{CH5}

\textbf{{DI}}: Command for identifying the device. It blinks a built-in LED 10 times.
It has no return.

\emph{Example}: {DI}
