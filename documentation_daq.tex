\chapter[DAQ Device Manual]{Python For The Lab {DAQ} Device Manual}\label{ch:pftl-daq-manual}
We believe that learning how to program software for a scientific laboratory can be achieved only through real-world examples. That is why the course was conceived around a small device that we are calling a General {DAQ} Device. In these pages, we document the behavior of the device. You will notice that it is similar to what you would typically find in the manual of any instrument in your lab.

\section{Capabilities}\label{sec:capabilities}
The \textbf{General {DAQ} Device} is a multi-purpose acquisition card that can handle digital and analog inputs and outputs. It runs on an {ARM} 32-bit microprocessor and derives its power from a {USB} connection. The device can handle one task per turn, but the outputs are persistent. It means that if you set the output of a particular port, it will remain constant until a command to change it is issued.

Normal Analog-to-Digital conversion times are in the order of 10 microseconds and are done with a resolution of 10 bits in the range $0-3.3\,\textrm{V}$. Digital-to-Analog conversions are done with a resolution of 12 bits in the range $0-3.3\,\textrm{V}$.

The {DAQ} possesses 2 Analog Output channels and 10 Analog Input channels. Each can be addressed independently; however, a degree of crosstalk can be observed, especially between neighboring ports. Sensitive applications would, therefore, need to use non-consecutive ports to mitigate this effect.

\section{Communication with a Computer}\label{sec:communication-with-acomputer}
The {DAQ} can communicate with the computer through a {USB} connection. However, the device has an on-board chip that converts the communication into serial. Therefore, it will appear listed as any other serial device.

The baud rate has to be set to 9600, and every command has to finish with the newline {ASCII} character. A newline character also terminates the messages generated by the device.

\section{List of Available Commands}\label{sec:list-of-commandsavailable}
\textbf{{IDN}}: Identifies the device; returns a string with information regarding the serial number and version of the firmware. \emph{Returns}: String with information

\textbf{{OUT}:}: Command for setting the output of an analog channel. It takes as arguments the channel \textbf{{CH}:\{\}} and the value, \textbf{\{\textless{}4\}}. The value has to be in the range $0-4095$, while the channel has to be either 0 or 1. \emph{Returns}: the value that the device received

\emph{Example}: {OUT}:{CH0}:1024

\textbf{{IN}}: Command for reading an analog input channel. It takes one argument, \textbf{{CH}:\{\}}, between 0 and 9. \emph{Returns}: integer in the range $0-1023$

\emph{Example}: {IN}:{CH5}
