\chapter{Writing the first Model for a Device}\label{chapter:device-model}

\section{Objectives}
In this chapter, we are going to develop a model for a real device. You are going to start seeing why the Model-View-Controller pattern is so useful. In the previous chapter, we discussed that models specify the ways in which the user can interact with devices. The logic of the experiment that we are going to perform is going to manifest in the models that we develop. 

Even if hard to realize at the beginning, most of the development time will be spent working and improving the models. Many controllers can be downloaded, installed, or quickly developed. The view, once developed, is going to stay more or less the same over long periods. Models, on the other hand, is where your true experiment is defined. In the models you are going to decide the order of the steps to perform an experiment, is where you are going to automatize tasks, etc. 

In this chapter, we are going to develop a model for the device based on the controller that we wrote on Chapter \ref{chapter:controller}. 

\section{Introduction}
The secret to successfully develop a model is to think beforehand what do we expect from the device, how are we actually planning to use it. The first time you start thinking about models it can be complicated because you need to anticipate your future needs. For example, the PFTL DAQ device doesn't handle units. It would be great if the model would allow us to specify the output voltage instead of converting it to an integer because it is what the driver uses. This requirement seems trivial and probably is the first one that comes to mind. Later, once you really start using the program, you will see that some other useful options were missing, and you could have saved a lot of time if you would have thought about them. 

Sadly, there is no magical recipe to teach you exactly how to develop models for your devices. The best we can do is to guide you through the processes of developing a model for the first time. With practice, you will get better at it, and you will learn how to anticipate your future needs. It is very important to separate what you expect from the device and what you expect from the experiment as a whole. If you master this technique, you will be able to use the models and controllers for very different purposes, without needing to re-write the entire program. 

In this book, we are going to see how to linearly change an analog output while recording an analog input. You know that the ultimate goal is to measure the I-V curve of a diode, but if you would hook the analog output to a piezo stage, you could be doing a scan in a confocal microscope. When the software is well designed, you will have plenty of smaller building blocks that you can keep using in your future projects. You won't need to write and think everything from scratch. Moreover, an improvement because of a new experiment can still benefit the program that controls your previous setup. 

All the predicates in this chapter rely on the fact that you really understand your own experiment before you start coding. Remember that what you are developing in this book is the simplest of the examples. In your own lab, there are going to be several devices, each will have a manual, they are going to be connected to each other. Be sure you understand not only the science behind what you are trying to measure but also the technical aspects of your setup, its limitations, and characteristics. 

\warning{It is impossible to overestimate the importance of reading the manuals of your devices. Hardware in the lab is not the same as consumer hardware. Things can break, signals may not make sense. Be always sure to understand the limits under which each component operates.}

\section{Base Model}\label{basemodel}
In practical terms, a model for a device is nothing more than a class. Remember that whenever a user is going to interact with the device, it is going to happen through the model and never directly to the controller (i.e. the driver). The first step is to think what methods we are going to need and, if possible, the output that each method is going to produce. If you haven't already, create a folder called \textbf{DAQ} inside the \emph{Model} folder and remember to also add an empty \texttt{\_\_init\_\_.py} file. 

We want all the DAQ models that we develop to have exactly the same methods. If we ever change the PFTL DAQ card for a different one, we would like our experiment to keep running. Therefore, the best strategy is to create a base class. This class will hold all the methods that are going to be used during our experiments. Create a file called \textbf{base.py} inside the DAQ folder, and define a class called \mintinline{python}{DAQBase}:

\begin{minted}{python}
class DAQBase:
    def __init__(self):
        pass
    
    def idn(self):
        pass
    
    def get_analog_value(self, channel):
        pass
    
    def set_analog_value(self, channel, value):
        pass
    
    def finalize(self):
        pass
\end{minted}

This class doesn't do anything by itself, it is only the schematics of what a model should contain. The \mintinline{python}{pass} that was added after every definition takes care of functions in which nothing happens. The step of defining a base class may not seem important at first, but it will become clearer later. The base class defines the common interface for all our devices. We have chosen to implement $5$ methods: An instantiation that will start the communication with our device. An identification, that will allow us to be sure which device we are communicating with. Two methods for setting and reading an analog channel. Pay attention to the fact that we define also the arguments needed. And lastly, a method to finalize the communication with the device.  

\section{Device Model}\label{devicemodel}
After defining what are the methods that a DAQ model should implement, we can start developing the true model for the {PFTL DAQ}. Let's start by creating a new file called \textbf{analog\_daq.py}. You can add the following:

\begin{minted}{python}
from PythonForTheLab.Controller.simple_daq import SimpleDaq 
from PythonForTheLab.Model.daq import DAQBase

class AnalogDaq(DAQBase):
    def __init__(self, port):
        self.port = port
\end{minted}

First, you have to notice that we are using absolute imports. We start with \mint{python}|from PythonForTheLab.Controller.simple_daq import SimpleDaq |, wich means that it is important that you added the root folder of your project to the PYTHONPATH as we discussed in the previous chapter. The \mintinline{python}{AnalogDaq} class inherits the \mintinline{python}{DAQBase} class, meaning that all the methods defined in it are going to be available in the new class. You can see this by running the following code:

\begin{minted}{python}
 from analog_daq import AnalogDaq
 daq = AnalogDaq('/dev/ttyACM0')
 print(daq.port)
 # /dev/ttyACM0
 daq.idn()
 # Nothing will happen, but it doesn't give an error
\end{minted}

The snippet above shows you that even if \mintinline{python}{AnalogDaq} doesn't define any methods besides \mintinline{python}{__init__}, you can use \mintinline{python}{daq.idn()} and it won't give you an error. This is happening, because when you inherit the \texttt{DAQBase} class, the methods defined in it are readily available in the child class. 

Note, however, that the \mintinline{python}{__init__} method takes \texttt{port} as an argument, while the DAQBase doesn't. In order to be consistent, you should modify DAQBase to reflect this need. Remember that the base and the model should always be synchronized, even if laziness dictates the contrary. 

\exercise{Change \texttt{DAQBase} in order to have an \texttt{\_\_init\_\_} method that takes the port as an argument}

To make our class actually do something, we need to add more code. For example, we need to start communicating with the device on the specified port. Let's assume that we want to start the communication right away when the class gets instantiated. We can do the following:

\begin{minted}{python}
class AnalogDaq(DAQBase):
    def __init__(self, port):
        super().__init__()
        self.port = port
        self.driver = SimpleDaq(self.port)
        self.driver.initialize()
\end{minted}

You can see that in the \mintinline{python}{__init__} method, we are instantiating the SimpleDaq with the specified port. You store the
object in an attribute called \mintinline{python}{driver}. The driver is going to handle all the communication with the device. Remember that the {MVC} pattern imposes that the user will never communicate with the device directly, but always through the model. If there is something that the device can do but the model can't, you should expand your model before bypassing it. Remember that when we developed the driver, we split the creation of the class and the beginning of the communication. 

\exercise{Add the \texttt{idn} method to your model in order to return the serial number from the device.}

The next step would be to be able to read a value from the device. We could add the following to the \texttt{AnalogDaq} model:

\begin{minted}{python}
def get_analog_value(self, port):
   value = self.driver.get_analog_value(port)
   return value
\end{minted}

Having a \emph{model} as a relay for methods defined in the \emph{controller} doesn't look too useful. So far, both the \texttt{idn} and \texttt{get\_analog\_value} methods do \emph{exactly} the same as the methods in the driver. However, this is exactly the point where things start to get interesting. The controller returns values that are simply integers, but at the beginning of the chapter, we said that we wanted to interact with the device in volts, including the values being read. 

If you go through the manual of the {PFTL DAQ}, you will see that it is possible to convert from the device integer value to volts by using the following relationship:

\begin{equation}
 \textrm{Volts} = 3.3\,\textrm{V} \cdot \frac{\textrm{bits}}{1023}
\end{equation}

\exercise{Complete the \mintinline{python}{get_analog_value} by converting the output from the integer value read generated by the device to the corresponding value in \emph{volts}.}

\exercise{Define the method \mintinline{python}{set_analog_value} taking into account that the value is going to be provided in volts. Remember that the maximum value is $4095$ instead of $1023$}. 

After completing the previous exercises, you can test that everything is working. Write a simple script that initializes the communication with the device, sets the output in some channel and reads some input. Verify that the values make sense when thinking in volts. 

\section{Adding real units to the code}\label{adding-units-to-thecode}
If you completed the previous exercises, you should have ended up with code that works great if you provide output voltages in volts and you know that the value you get when reading an analog input is also in volts. Volts seem like a natural choice in the context in which we are working now, but you could have also decided to use millivolts, for example. Handling units in computer programs is always a headache because you need to document exactly what units each function takes and returns. 

If you are hesitant about the impact that different unit systems can have, there is a great example involving a multi-million dollar satellite. You can check the \href{http://articles.latimes.com/1999/oct/01/news/mn-17288}{news article} with the story of how the Mars Climate Orbiter fell from its orbit because engineers from the US failed at using the established units of measure in their software, resulting in a mix of metric and imperial systems. 

Fortunately for us, there is a Python package called \emph{Pint}, that allows us to work with \emph{real} units in our programs. Let's quickly see how Pint can be used in our programs with a simple example:

\begin{minted}{pycon}
>>> import pint

>>> ur = pint.UnitRegistry()
>>> meter = ur('meter')
>>> b = 5*meter
>>> print(b)
5 meter
>>> c = b.to('inch')
>>> print(c)
196.8503937007874 inch
\end{minted}

In the example above, you can start to see the true potential of \emph{Pint}. First, we import the package and start the \emph{Unit
Registry}. In principle, Pint allows you to work with custom-made units, but the fundamental ones are already included in their own \emph{Unit Registry}. Then, because of convenience, we define the variable \texttt{meter}, as actually the unit meter. Finally, we assign the value of \texttt{5 meters} to a variable called \texttt{b}. In this case, \texttt{b} is of type \texttt{Quantity}, therefore it is not just a number, but a number and a unit attached to it. \emph{Pint} allows you to convert between units, and this is how we create the variable \texttt{c}, which is 5 meters converted to inches. And things can get very interesting:

\begin{minted}{pycon}
 >>> b == c
 True
\end{minted}

Even if the numeric value of \texttt{b} and \texttt{c} is different, they are still equal to each other, exactly as you would have imagined. But this is not all. Look at the following example:

\begin{minted}{pycon}
>>> d = c*b
>>> print(d.to('m**2'))
25.0 meter ** 2
>>> print(d.to('in**2'))
38750.07750015501 inch ** 2
>>> t = 2.5*ur('s')
>>> v = c/t
>>> print(v.to('in/s'))
78.74015748031496 inch / second
\end{minted}

You can see in the example above that \emph{Pint} can handle combined units such as what happens with speed. You can operate on them as you would with pen and paper, and you can convert from one to another very easily. Pint also supports conversion between units that have a complex relationship. For example:

\begin{minted}{pycon}
>>> current = 5*ur('A')
>>> res = 10*ur('ohm')
>>> voltage = current*res
>>> print(voltage)
50 ampere * ohm
>>> print(voltage.to('V'))
50.0 volt
>>> print(voltage.m_as('mV'))
50000.0
\end{minted}

The snippet above shows you that Pint is able to understand the relationship between Amperes, Ohms, and Volts. There is one more feature that is important to point out: Pint is able to parse strings in order to separate the units from the numbers. For example, we can do the following:

\begin{minted}{pycon}
 >>> current = ur('5 A')
 >>> resistance = ur('10 ohm')
\end{minted}

Being able to parse strings so easily is going to make our life much easier when we will be dealing with user input. Now you have seen how to handle \emph{real} units on your code. However, the device still requires you to set the output using plain numbers. In the context of Pint, just the number, without the units is called the \emph{magnitude}. To get the magnitude out of a quantity, you can do the following:

\begin{minted}{pycon}
 >>> current = ur('5 A')
 >>> current_mag = current.m
 >>> print(current)
 5 ampere
 >>> print(current_mag)
 5
 >>> current_ma = current.m_as('mA')
 >>> print(current_ma)
 5000.0
\end{minted}

We start with a quantity called \texttt{current} of $5\,\textrm{A}$. If we just append the \texttt{.m} to the variable, we get the magnitude in whatever unit it is already expressed. If we want to be sure to get the magnitude in a specific unit, we use the command \texttt{.m\_as()}. In our case, we will need to transform the user input to an integer, and we will not need to assume it is in volts, we can transform it to volts before converting it to an integer. The \texttt{set\_analog\_value} method would look like this:

\begin{minted}{python}
 def set_analog_value(self, channel, value):
     value_volts = value.m_as('V')
     value_int = round(value_volts/3.3*4095)
     self.driver.set_analog_value(channel, value_int)
\end{minted}

You can see that we first transform the value to volts and get only the magnitude. Then we transform that value to bits, using the round function to get a true integer after the operation. We use that rounded value to set the output on the device. You can see that our program is now very flexible since the user can provide the output value in whatever units she pleases, provided that they can be transformed to volts.

\exercise{Update the method \texttt{get\_analog\_value} so it generates an in volts. Pay attention to the fact that you will need to import pint and create the unit registry before you define your class in order to be able to use it.}

\section{Testing the DAQ Model}
At this point, we have a very functional program. We are able to handle units, we have split the logic of the units from the driver, meaning that we can easily share our code with colleagues or the world. It is time now to test our program. On the root folder of your project, outside of the PythonForTheLab folder, create a file called \textbf{test\_daq.py}. Add the following code to it:

\begin{minted}{python}
import numpy as np
import pint

from PythonForTheLab.Model.DAQ.analog_daq import AnalogDaq

ur = pint.UnitRegistry()
V = ur('V')

daq = AnalogDaq('/dev/ttyACM0') # <-- Remember to change the port
print('The DAQ serial number is {}'.format(daq.idn()))

# 20 Values with units in a numpy array
volt_range = np.linspace(0, 3, 20) * V 
current = [] # Empty list to store the values

for volt in volt_range:
    daq.set_analog_value(0, volt)
    current.append(daq.get_analog_value(0))

print(current)
\end{minted}

At this stage, you should be able to go through the code above yourself, it should be straightforward to understand. The first thing that should grab your attention is that we are storing information into a variable called \texttt{current}, while what we are measuring is actually a voltage. We should transform the voltage measured to a current. Following Ohm's law, we can divide the voltage by the resistance used in our experiment. We can try to do that:

\begin{minted}{python}
 resistance = ur('220ohm')
 voltage = daq.get_analog_value(0)
 current = voltage/resistance
\end{minted}

If you run the code above, you will get the following error:

\begin{minted}{python}
 [...]
 ValueError: Cannot operate with Quantity and Quantity of different registries.
\end{minted}

The error is very descriptive, but it is hard to understand what is really happening and how to solve it. What you have to remember is that the unit registry is a collection of rules that allow you to transform from one quantity to another. Pint is flexible enough to allow you to define your own rules, your own units. The problem is that you can't convert units across unit registries. So, it all comes down to using only one registry throughout all your program. In your code, if instead of importing pint and creating a new unit registry you do the following:

\begin{minted}{python}
 from PythonForTheLab.Model.DAQ.simple_daq import ur
\end{minted}

Then, you would be using the same unit registry that you defined in order to make the \texttt{get\_analog\_value} method work with units. Go ahead and try it. However, importing the unit registry from the DAQ model doesn't seem the most appropriate. In the previous chapter, we have seen how importing works in Python (section \ref{section:importing-python}). We have also seen that you can include code into the \texttt{\_\_init\_\_.py} files. Therefore, if we would include a unit registry into the init file of the PythonForTheLab folder, it would become very easy to import elsewhere. 

Open the file \textbf{PythonForTheLab/\_\_init\_\_.py} and add the following code:

\begin{minted}{python}
 import pint
 
 ur = pint.UnitRegistry()
\end{minted}

Every time you would like to use units and the unit registry, you will be able to do the following:

\begin{minted}{python}
 from PythonForTheLab import ur
\end{minted}

\exercise{Improve DAQ model in order to use the main unit registry and not one defined locally.}

\exercise{Modify the example that we developed for testing the DAQ model so it uses the main unit registry, and see that it works as expected.}

The DAQ model is almost complete, but there is still one detail missing. If you run the following code (assuming we are still building on top of the example script developed a bit earlier):

\begin{minted}{python}
 print(daq.idn())
\end{minted}

You will see that nothing happens. The \texttt{idn} method is not returning the serial number of the device. What should surprise you is that is not throwing an error either. Remember that we used the base DAQ class as the parent for the model. This means that the method is defined, but it doesn't do anything. It will be your task to solve this problem:

\exercise{Define a method called \texttt{idn} in the DAQ model in order to return the real serial number from the device.}

\section{Conclusions}\label{conclusions}
This chapter is focused on showing you what does \emph{Model} mean in the {MVC} pattern when developing software for the lab. Of course, the examples proposed in this book are very simple and sometimes the gain is not going to be evident. However, as soon as you start building more complex applications, you will see the benefits of splitting the Model from the Controller and, later on, the View from the Model. 

You have built two classes, a base class that helps you keep track of the methods and properties that each device is going to have. The second class inherits from the base class. Right now, this could have been avoided because there is no benefit in having a base class. However, we will see in the following chapter that it can save you a lot of time, plus is a nice way of being organized with your code. Remember that most of the things that we teach in the book can be regarded as best practices. In the end, is up to you to decide how much you are willing to follow. 

In this chapter, you have also started to use more complex options of classes in Python. If you manage to understand what is inheritance and how it can benefit your development, you are already a long way ahead. Object-oriented programming is a very broad topic that requires patience and practice to master, but it is nothing that can't be understood quickly with the proper examples. We are going to see more about inheritance and classes when we deal with the view. 

This chapter has a lot of exercises for you to complete. If you get lost, or just don't want to code yourself, remember that the full program is available on \href{https://github.com/PFTL/SimpleDaq}{Github}\footnote{https://github.com/PFTL/SimpleDaq}.
