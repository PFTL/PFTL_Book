\chapter{Writing the first Model for a Device}\label{writing-the-first-model-for-adevice}

\section{Objectives}
In this chapter, you are going to develop a model for a real device. You
are going to start seeing why the Model-View-Controller pattern is so
useful. Models are specific ways in which the user can interact with the
devices or with other data sources. In the models is where you are going
to start developing the logic of your experiment, where you are going to
establish the safeguards to avoid things going wrong.

Even if hard to believe, you should spend the most of your time working
and improving the models. The controllers can be downloaded or quickly
developed, the view is more often than not a simple script to read a
file, or you can use existing libraries. The Model, however, is where
your true experiment is defined. Is where you have to be very focused
and decide in which order the steps are taken and what things can be
given for granted or not.

In this chapter you are going to develop the core of the program, and
hopefully, the basis for your future program when you are working in your
own lab.

\section{Introduction}
When developing models, you have to keep in mind that you will start
specifying how the devices are going to behave. This means that models
are somewhat specific to your experiment and sometimes the names you
give to the functions can reflect a specific task, but it doesn't mean
that it cannot be used in a different context. For example, a confocal
microscope normally needs a couple of analog outputs to perform a scan,
and at least one input to record a signal. The analog outputs can be
plugged into a piezo device that moves either the sample or the beam. If
you do this systematically, you can construct an image pixel by pixel.
You move to a location, record the input, move to a new location, etc.
If you want to study the current that flows through a diode as a function of the voltage applied, you could use the same code developed
for the confocal microscope without changing a single line. However,
using a function called \mintinline{python}{do_confocal_scan} to measure a current
through a diode may be misleading.

Developing a model for the first time is a bit tricky because it implies
that you know what is going to happen in the future. In a normal
situation, you will have a lot of back and forths with the models, but
for the purposes of this book, it is important that you follow the
recommendations carefully. For developing a model, you need to
understand how your experiment works and what do you want to achieve.
The goal of this book is to measure the current that flows through a
diode. But if you are building a confocal microscope, for example, you
should really understand which analog outputs you are using, how your
piezo is connected, what are the limits of the device, etc.

It is impossible to overstress how important it is for you to understand
your own experiment before starting to code. Remember that what you are
developing in this book is the simplest of the examples. Nothing can
really be broken, no expensive device is going to receive an input
voltage higher than what it can handle. What you probably have in your
setup is much more complex. Maybe you have to use external triggers,
maybe you have to push some device to the edge of its capabilities and
you should be aware of what the limits are. There is no better friend in
the lab than the manuals and the technical support.

\section{Base Model}\label{basemodel}
A model in Python is a class with some special elements. The first step
when building a model is to think what methods we are going to need, and
if possible what is the output that each method is going to produce. A
method may produce an array of values, a single value, no value, etc.
Remember that the idea of having models, amongst other things, is that
if in the future you change the device, you only need to write a new
model with the same functions and the rest of your code will work
without further problems. Since we are writing a model for a {DAQ} (an
acquisition card), we should create an appropriate package in the
\emph{Model} folder. Create a folder called \emph{daq} and remember to
add an empty \mintinline{python}{__init__.py} file in each new folder you make.
Perhaps you already did this in the previous chapter.

Instead of starting with the model for our {DAQ}, you can instead create
a base class, in which you will keep track of all the functions that a proper model should have. You can create a file called \textbf{base.py}
and define a class called \mintinline{python}{DAQBase}:

\begin{minted}{python}
class DAQBase(object):
    def __init__(self):
        pass
    
    def idn(self):
        pass
    
    def get_analog_value(self, channel):
        pass
    
    def set_analog_value(self, channel, value):
        pass
    
    def get_digital_value(self, channel):
        pass
    
    def set_digital_value(self, channel, value):
        pass
\end{minted}

This class doesn't do anything, it only defines the methods that you
need to develop for each device. The \mintinline{python}{pass} takes care of
functions in which nothing happens and nothing is returned either.
Having the base class defined allows you to do several things. One is to
keep track of all the methods that you need to define when you develop a new model, as we are going to do later. The other is that you can
inherit this class, and be sure that all the functions are already
present, even if they don't do anything.

Each method is quite self-explanatory. The \mintinline{python}{idn} refers just to
identify the device. You also see that setting and getting values
require a port, and a value. You can also start to notice that for this
simple example, the model is only a relay for the Controller class. For
the time being, there is no method in the base class that doesn't exist
in the controller itself.

\section{Device Model}\label{devicemodel}
Once you have established what are the base methods that your model
needs to define, you can start developing a model for the real device.
There are two important modules that you will need to import, the base
model and the controller for your device. Create a file called
\textbf{analog\_daq.py} and add the following:

\begin{minted}{python}
from ...Controller import SimpleDaq
from .base import DAQBase

class AnalogDaq(DAQBase):
    def __init__(self, port):
        super().__init__()
        self.port = port
\end{minted}

First, you have to notice that you are using relative imports, i.e., the
three \mintinline{python}{...} before \mintinline{python}{Controller}. This means that Python
will look three directories up for a module called \emph{Controller}.
One dot is for the current directory, two brings you to the \emph{Model}
and three to the root of the package. When you are not sure whether the
package you are developing is in the \emph{path} or not, using relative
imports is a way of ensuring that you can still find what you need to
import. If the package is in the path, you could change the first
line to:

\begin{minted}{python}
from PythonForTheLab.Controller import SimpleDaq
\end{minted}

Going back to the code, the \mintinline{python}{AnalogDaq} class inherits the
\mintinline{python}{DAQBase} class, meaning that all the methods and properties
defined in it are going to be available also in the new class. When you
inherit a class, you need to be sure that the \mintinline{python}{__init__}
method of the parent is executed. This is what the line
\mintinline{python}{super().__init__()} is doing. We request a port from the user, that for the time being is just stored in a property called
\mintinline{python}{port}. You can use the class, and you will see that the methods
defined in \mintinline{python}{DAQBase} are available also in \mintinline{python}{AnalogDaq},
even if they don't do anything:

\begin{minted}{pycon}
>>> from analog_daq import AnalogDaq
>>> dev = AnalogDaq('COM1')
>>> dev.idn()
\end{minted}

To make the class actually work, you need to start overwriting the
methods defined in the base class. This just means defining a new method
with the same name. You can start with the \mintinline{python}{initialize} to open
the communication with the device.

\begin{minted}{python}
class AnalogDaq(DAQBase):
    def __init__(self, port):
        super().__init__()
        self.port = port
        self._driver = SimpleDaq(self.port)
    
    def initialize(self):
        self._driver.initialize()
\end{minted}

You can see that in the \mintinline{python}{__init__} method, you are
instantiating the SimpleDaq with the specified port. You store the
object in a property called \mintinline{python}{_driver}. The driver is going to
handle all the communication with the device. Remember that the {MVC}
pattern imposes that the user will never communicate with the device
directly, but always through the model. If there is something that the
device can do but the model can't, you should expand your model before
bypassing it.

Python doesn't have a way of declaring private properties as other
languages do. Private properties are those properties that can be used
only within an object, but not from outside. After you instantiate the
class, you can get access to the driver just by typing
\mintinline{python}{dev._driver}. The fact that the property starts with an
underscore is a sign that it should be left alone and not used from
outside, but there is no easy way to enforce this in Python. The first
one that has to be consistent with its use is yourself.

The \mintinline{python}{initialize} method uses the driver to initialize the
communication with the device, using the specified port. Probably you
have realized it already, the \mintinline{python}{DAQBase} class doesn't have an
\mintinline{python}{initialize} method defined. Don't worry about it now, but you
will see why you should have added it also there. Now it is time to
improve the class, for example by defining a \mintinline{python}{get_analog_value}
method.

\begin{minted}{python}
def get_analog_value(self, port):
   value = self._driver.get_analog_value(port)
   return value
\end{minted}

Having a \emph{model} as a relay for methods defined in the
\emph{controller} doesn't look too useful. But now is when things start
to get interesting. The controller returns arguments that are simply
integers, but that you can translate to a voltage, as specified in the
device manual. As you can imagine, improving the rest of the class is up
to you.

\exercise{Complete the \mintinline{python}{get_analog_value} by setting the proper units to the
returned value. Don't forget to use Pint.}

\exercise{Overwrite the method \mintinline{python}{set_analog_value} from the base class to
actually communicate with the device. Use units! Think about the limits,
and what happens if a user sends a plain number instead of a
\emph{Quantity}}

After you have defined the methods, you can test that everything is
working. You can write a simple script that initializes the
communication with the device, sends some outputs and reads some inputs.
Create a file called \textbf{test\_daq.py} and write something like the
code below:

\begin{minted}{python}
import numpy as np
import pint

from analog_daq import AnalogDaq

ur = pint.UnitRegistry()
V = ur('V')

daq = AnalogDaq('/dev/ttyACM0') # <-- Remember to change the port
daq.initialize()
print('The DAQ serial number is {}'.format(daq.idn()))

# 20 Values with units in a numpy array
volt_range = np.linspace(0, 3, 20) * V 
current = [] # Empty list to store the values

for volt in volt_range:
    daq.set_analog_value(0, volt)
    current.append(daq.get_analog_value(0))

print(current)
\end{minted}

At this stage, you should be able to go through the code yourself. It is
not very handy because if you need to change which port you are using
you need to go through the code, or if the device is connected to a
different port. However, you can see that it is very simple to perform a
measurement. If you would be working with a confocal microscope, you
could have done basically the same. Probably you have noticed that when
you try to get the serial number of your device nothing happens. This is
because you didn't define an \mintinline{python}{idn()} method in your class, but it
is important to point out that there were no errors because the method
exists in the class, inherited from the base.

\exercise{Fix the \mintinline{python}{idn()} method to display the correct information.}

\section{Dummy DAQ}\label{dummy-daq}
You have just seen how to develop a model for a device such as the
{PFTL} {DAQ}. However, when you develop software for the lab, sometimes
you would like to test your program without having the real device
connected to the computer. It can be a safety precaution or simply
because you are not working on the lab computer. To achieve this, you
can define a fake model that generates data without interacting with a
device. You can call these devices whatever you like, in this case, we
use \textbf{dummy}. The code will look like this:

\begin{minted}{python}
from .base import DAQBase

class DummyDaq(DAQBase):
    serial_number = '1234ABC'
    
    def __init__(self):
        super().__init__()
\end{minted}

You start by importing the base class \mintinline{python}{DAQBase}, as you did for
the real \emph{model}, but you don't need the controller since you are
not going to communicate with any device. The class defines a fake
serial, just to be able to return it as if it was the real device. Now
you have to think what do you actually expect from a fake device. Since
you are developing a dummy {DAQ}, you don't need to do anything when
setting a value to an output, but you do need to get a value when you
read back. And you need to use the proper units as well. It can look
something like this:

\begin{minted}{python}
import numpy as np
import pint

from .base import DAQBase

ur = pint.UnitRegistry()
V = ur('V')

class DummyDaq(DAQBase):
    serial_number = '1234ABC'
    
    def __init__(self):
        super().__init__()
    
    def idn(self):
        return self.serial_number
    
    def get_analog_value(self, port):
        return np.random.random(1)*V
\end{minted}

The \mintinline{python}{idn} is quite clear, it just returns the serial number that
was defined during initialization. The \mintinline{python}{get_analog_value}
accepts a port as a parameter, even if they don't use it, but you need
to keep the compatibility with what was already defined. The value that
is returned is a random number with volts as units. Remember that the
random generator outputs values only between 0 and 1. You can try to run
the test, but using this DummyClass instead. Open the file
\textbf{test\_daq.py}, and replace the line:

\begin{minted}{python}
# from analog_daq import AnalogDaq
from dummy_daq import DummyDaq as AnalogDaq
\end{minted}

And run \textbf{test\_daq.py}. You will see that an error appears
because you are using the \mintinline{python}{initialize} method in the test, but it
is not defined within the class. This happened because you included
\mintinline{python}{initialize} in the
\mintinline{python}{AnalogDaq} but you never added it to the DAQBase.
Now you see the power that being organized and systematic has. Keeping
track of the methods in a general, base class, allows you not to make
the same mistakes. Moreover, it would have allowed you to run the test
without issues.

\exercise{Add an \mintinline{python}{initialize} method into the base class.}

If you run the test after completing the exercise, you will see that
there are no errors appearing, even if you don't modify the dummy class.
Now you see that you can exchange the models in a relatively simple way,
and the script that is responsible for measuring works without problems.
Of course, this is a simple example, where you exchange a real device
for a dummy one, but if you would have two different devices, exchanging
them would have been just as simple and the measurement would have
been performed.

\exercise{Make the \mintinline{python}{get_analog_value} return the value of a sine function
and not just a random value.}

\section{Conclusions}\label{conclusions}
This chapter is focused on showing you what does \emph{Model} mean in
the {MVC} pattern when developing software for the lab. Of course, the
examples proposed in this book are very simple and sometimes the gain is
not going to be apparent. However, as soon as you start building more
complex applications, you will see the benefits of splitting the Model
from the Controller, and later on, the View from the Model.

You have built three classes, a base class that helps you keep track of
the methods and properties that each device is going to have. Keeping
this class up to date and clean is very useful because it allows you and
other developers to have a clear starting point. You have seen that by
skipping some steps you run into errors, that at this moment are easy to
debug and solve, but later they may become more obscure, depending on
what triggered them.

In this chapter, you have also started to use more complex options of
classes in Python. Hopefully, you can understand what is going on and
learn from the examples. Once you see the potential that defining
classes have, you will never stop thinking about them. This has also
been a chapter where you had to code a lot yourself, making your own
decisions. If you ever get lost or think that you are departing too much
from what we are asking, you can always check the GitHub repository of
the book, where you can find the code for every chapter.
