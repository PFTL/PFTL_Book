\chapter{Classes in Python}\label{ch:classes-in-python}
Python is an object-oriented programming (OOP) language. Object-oriented programming is a programming design that allows developers to define not only the type of data of a variable but also the operations that can act on that data. For example, a variable can be of type integer, float, or string. We know that we can multiply an integer to another, or divide a float by another, but that we cannot add an integer to a string. Objects allow programmers to define operations both between different objects as with themselves. For example, we can define an object \mintinline{python}{person}, add birthday, and have a function that returns the person's age.

In the beginning, it is not clear why objects are useful, but over time it becomes impossible not to think with objects in mind. Python takes the ideas of objects one step further and considers every variable an object. Even if you didn't realize, you might have already encountered some of these ideas when working with numpy arrays, for example. In this chapter, we are going to cover from the very basics of object design to slightly more advanced topics in which we can define custom behavior for most of the common operations.

\section{Defining a Class}\label{sec:defining-a-class}
Let's dive straight into how to work with classes in PythonPython. Defining a class is as simple as doing:

\begin{minted}{python}
class Person():
    pass
\end{minted}

When speaking, it is tough not to interchange the words \mintinline{python}{Class} and \mintinline{python}{Object}. The reality is that the difference between them is very subtle: an object is an instance of a class. It means that we use classes when referring to the type of variable, while we use object to the variable itself. It is going to become more apparent later on.

In the example above, we've defined a class called \mintinline{python}{Person} that doesn't do anything (that is why it says \mintinline{python}{pass}.) We can add more functionality to this class by declaring a function that belongs to it. Create a file called \textbf{person.py} and add the following code to it:

\begin{minted}{python}
class Person():
    def echo_name(self, name):
        return name
\end{minted}

In Python, the functions that belong to classes are called \textbf{methods}. For using the class, we have to create a variable of type person. Back in the Python Interactive Console, you can, for example, do:

\begin{minted}{pycon}
>>> from person import Person
>>> me = Person()
>>> me.echo_name("John Snow")
\end{minted}

The first line imports the code into the interactive console. For this to work, you must start Python from the same folder where the file \textbf{person.py} is located. When you run the code above, you should see as output \mintinline{python}{John Snow}. We omitted an important detail: the presence of \mintinline{python}{self} in the declaration of the method. All the methods in Python take a first input variable called self, referring to the class itself. For the time being, don't stress yourself about it, but bear in mind that when you define a new method, you should always include the \mintinline{python}{self}, but when calling the method, you should never include it. You can also write methods that don't take any input, but still have the \mintinline{python}{self} in them, for example:

\begin{minted}{python}
def echo_True(self):
    return "True"
\end{minted}

that can be used by doing:

\begin{minted}{pycon}
>>> me.echo_True()
\end{minted}

So far, defining a function within a class has no advantage at all. The main difference and the point where methods become handy is because they have access to all the information stored within the object itself. The \mintinline{python}{self} argument that we are passing as the first argument of the function is exactly that. For example, we can add the following two methods to our class Person:

\begin{minted}{python}
def store_name(self, name):
    self.stored_name = name

def get_name(self):
    return self.stored_name
\end{minted}

And then we can execute this:

\begin{minted}{pycon}
>>> me = Person()
>>> me.store_name('John Snow')
>>> print(me.get_name())
>>> print(me.stored_name)
\end{minted}

What you can see in this example is that the method \mintinline{python}{store_name} takes one argument, \mintinline{python}{name} and stores it into the class variable \mintinline{python}{stored_name}. As with methods, variables are called \textbf{properties} in the context of a class. The method \mintinline{python}{get_name} just returns the stored property. What we show in the last line is that we can access the property directly, without the need to call the \mintinline{python}{get_name} method. In the same way, we don't need to use the \mintinline{python}{store_name} method if we do:

\begin{minted}{pycon}
>>> me.stored_name = 'Jane Doe'
>>> print(me.get_name())
\end{minted}

One of the advantages of the attributes of classes is that they can be of any type, even other classes. Imagine that you have acquired a time trace of an analog sensor, and you had also recorded the temperature of the room when the measurement started. You can easily store that information in an object:

%! Suppress = Ellipsis
\begin{minted}{python}
measurement.temperature = '20 degrees'
measurement.timetrace = np.array([...])
\end{minted}

What you have so far is a vague idea of how classes behave, and maybe you are starting to imagine some places where you can use a class to make your daily life easier and your code more reusable. However, this is just the tip of the iceberg. Classes are potent tools.

\section{Initializing Classes}\label{sec:initializing-classes}
Instantiating a class is the moment in which we call the class and pass it to a variable. In the previous example, the instantiation of the class happened at the line reading \mintinline{python}{me = Person()}. You may have noticed that the property \mintinline{python}{stored_name} does not exist in the object until we assign a value to it. It can give severe headaches if someone calls the method \mintinline{python}{get_name} before actually having a name stored (you can give it a try to see what happens!). Therefore it is handy to run a default method when the class is first called. This method is called \mintinline{python}{__init__}, and you can use it like this:

%! Suppress = Ellipsis
\begin{minted}{python}
class Person():
    def __init__(self):
        self.stored_name = ""

    [...]
\end{minted}

If you go ahead and run the \mintinline{python}{get_name} without actually storing a name beforehand, now there is no error, it just returns an empty string. While initializing, you can also force the execution of other methods, for example:

\begin{minted}{python}
def __init__(self):
    self.store_name('')
\end{minted}

It has the same final effect. It is common (and smart) practice, to declare all the variables of your class at the beginning, inside your \mintinline{python}{__init__}. In this way, you don't depend on calling specific methods to create the variables.

As with any other method, you can have an \mintinline{python}{__init__} method with more arguments than just \mintinline{python}{self}. For example you can define it like this:

\begin{minted}{python}
def __init__(self, name):
    self.stored_name = name
\end{minted}

Now the way you instantiate the class is different, you will have to do it like this:

\begin{minted}{python}
me = Person('John Snow')
print(me.get_name())
\end{minted}

When you do this, your previous code stops working, because now you have to set the \mintinline{python}{name} explicitly. If there is any other code that does \mintinline{python}{Person()} fails. The proper way of altering the functioning of a method is to add a default value in case no explicit value is passed. The \py{__init__} would become:

\begin{minted}{python}
def __init__(self, name=''):
    self.stored_name = name
\end{minted}

With this modification, if you don't explicitly specify a name when instantiating the class, it defaults to \py{''}, i.e., an empty string.

\questionInfo{Exercise}{Improve the \mintinline{python}{get_name} method to print a warning message in case the name was not set}

\section{Defining Class Properties}\label{sec:defining-class-properties}
So far, if you wanted to have properties available right after the instantiation of a class, you had to include them in the \py{__init__} method. However, this is not the only possibility. You can define properties that belong to the class itself. Doing it is as simple as declaring them before the \py{__init__} method. For example, we could do this:

%! Suppress = Ellipsis
\begin{minted}{python}
class Person():
    birthday = '2010-10-10'
    def __init__(self, name=''):
        [...]
\end{minted}

If you use the new \py{Person} class, you will have a property called \py{birthday} available, but with some interesting behavior. Let's see. First, let's start as always:

\begin{minted}{pycon}
>>> from person import Person
>>> guy = Person('John Snow')
>>> print(guy.birthday)
2010-10-10
\end{minted}

What you see above is that it doesn't matter if you define the birthday within the \py{__init__} method or before, when you instantiate the class, you access the property in the same way. The main difference is what happens before instantiating the class:

\begin{minted}{pycon}
>>> from person import Person
>>> print(Person.birthday)
2010-10-10
>>> Person.birthday = '2011-11-11'
>>> new_guy = Person('Cersei Lannister')
>>> print(new_guy.birthday)
2011-11-11
\end{minted}

What you can see in the code above is that you can access class properties before you instantiate anything. That is why they are class and not object properties. Subtleties apart, once you change the class property, in the example above the birthday, next time we create an object with that class, it receives the new property. In the beginning, it is hard to understand why it is useful, but one day you need it, and it saves you plenty of time.

\section{Understanding Inheritance}\label{sec:inheritance}
One of the advantages of working with classes in PythonPython is that it allows you to use the code from other developers and expand or change its behavior without modifying the original code. The best would be to see it in action. So far, we have a class called \py{Person}, which is generally but not too useful. Let's assume we want to define a new class, called \py{Teacher}, that has the same properties as a \py{Person} (i.e., name and birthday) plus it can teach a class. You can add the following code to the file \textbf{person.py}:

\begin{minted}{python}
class Teacher(Person):
    def __init__(self, course):
        self.course = course

    def get_course(self):
        return self.course

    def set_course(self, new_course):
        self.course = new_course
\end{minted}

Note that in the definition of the new \py{Teacher} class, we have added already \py{Person}. In Python jargon, this means that the class \py{Teacher} is a child of the class \py{Person}, or viceversa, that \py{Person} is the parent of \py{Teacher}. This is called \textbf{inheritance} and is not only very common in Python programs, it is one of the characteristics that makes Python so versatile. You can use the class \py{Teacher} in the same way as you have used the class \py{Person}:

\begin{minted}{pycon}
>>> from person import Teacher
>>> me = Teacher('math')
>>> print(me.get_course)
math
>>> print(me.birthday)
2010-10-10
\end{minted}

However, if you try to use the teacher's name it is going to fail:

%! Suppress = Ellipsis
\begin{minted}{pycon}
>>> print(me.get_name())
[...]
AttributeError: 'Teacher' object has no attribute 'stored_name'
\end{minted}

The reason behind this error is that \py{get_name} returns \py{stored_name} in the class Person. However, the property \py{stored_name} is created when running the \py{__init__} method of Person, which didn't happen. You could have changed the code above slightly to make it work:

%! Suppress = SentenceEndWithCapital
\begin{minted}{pycon}
>>> from person import Teacher
>>> me = Teacher('math')
>>> me.store_name('J.J.R.T.')
>>> print(me.get_course)
math
>>> print(me.get_name())
J.J.R.T.
\end{minted}

However, there is also another approach to avoid the error. You could simply run the \py{__init__} method of the parent class (i.e. the base class), you need to add the follwing:

%! Suppress = Ellipsis
\begin{minted}{python}
class Teacher(Person):
    def __init__(self, course):
        super().__init__()
        self.course = course
    [...]
\end{minted}

When you use \py{super()}, you are going to have access directly to the class from which you are inheriting. In the example above, you explicitly called the \py{__init__} method of the parent class. If you try again to run the method \py{me.get_name()}, you see that no error appears, but also that nothing appears on the screen. This is because you triggered the \py{super().__init__()} without any arguments, and therefore the name defaulted to the empty string.

\questionInfo{Exercise}{Improve the \py{Teacher} class to be able to specify a name to it when instantiating, for example, you would like to do this: \mintinline{pycon}{Teacher('Red Sparrow', 'Math')}}

\section{Looking into the Finer Details of Classes}\label{sec:finer-details-of-classes}
With what you have learned up to here, you can achieve many things. It is just a matter of thinking about how to connect different methods when it is useful to inherit. Without a doubt, it helps you to understand the code developed by others. There are, however, some details that are worth mentioning, because you can improve how your classes look and behave.

\subsection{Printing objects}\label{subsec:printing-objects}
Let's see, for example, what happens if you print an object:
\begin{minted}{pycon}
>>> from person import Person
>>> guy = Person('John Snow')
>>> print(guy)
<__main__.Student object at 0x7f0fcd52c7b8>
\end{minted}
The output of printing \py{guy} is quite ugly and is not particularly useful. Fortunately, you can control what appears on the screen. You have to update the \py{Person} class. Add the following method to the end:

\begin{minted}{python}
def __str__(self):
    return "Person class with name {}".format(self.stored_name)
\end{minted}

If you run the code above, you get the following:
\begin{minted}{pycon}
>>> print(guy)
Person class with name John Snow
\end{minted}

You can get very creative. It is also important to point out that the method \py{__str__} is used when you want to transform an object into a string, for example like this:

\begin{minted}{pycon}
>>> class_str = str(guy)
>>> print(class_str)
Person class with name John Snow
\end{minted}

Which also works if you do this:

\begin{minted}{pycon}
>>> print('My class is {}.'.format(guy))
\end{minted}

Something important to point out is that this method is inherited. Therefore, if instead of printing a \py{Person}, you print a \py{Student}, you see the same output, which may or may not be the desired behavior.

\subsection{Defining complex properties}\label{subsec:defining-complex-properties}
When you are developing multiple classes, sometimes you would like to alter the behavior of assigning values to an attribute. For example, you would like to change the age of a person when you store the year of birth:
\begin{minted}{pycon}
>>> person.year_of_birth = 1980
>>> print(person.age)
38
\end{minted}

There is a way of doing this in Python which can be easily implemented even if you don't fully understand the syntax. Working again in the class \py{Person}, we can do the following:
\begin{minted}{python}
class Person():
    def __init__(self, name=None):
        self.stored_name = name
        self._year_of_birth = 0
        self.age = 0

    @property
    def year_of_birth(self):
        return self._year_of_birth

    @year_of_birth.setter
    def year_of_birth(self, year)
        self.age = 2018 - year
        self._year_of_birth = year
\end{minted}
Which can be used like this:
\begin{minted}{pycon}
>>> from people import Person
>>> me = Person('Me')
>>> me.age
0
>>> me.year_of_birth = 1980
>>> me.age
32
\end{minted}

Python gives you control over everything, including what does the \py{=} do when you assign a value to an attribute of a class. The first time you create a \py{@property}, you need to specify a function that returns a value. In the case above, we are returning \py{self._year_of_birth}. Just doing that allows you to use \py{me.year_of_birth} as an attribute, but it fails if you try to change its value. It is called a read-only property. If you are working in the lab, it is useful to define methods as read-only properties when you can't change the value. For example, a method for reading the serial number would be read-only.

If you want to change the value of a property, you have to define a new method. This method is going to be called a \textit{setter}. That is why you can see the line \py{@year_of_birth.setter}. The method takes an argument that triggers two actions. On the one hand, it updates the age; on the other, it stores the year in an attribute. It takes a while to get used to, but it can be convenient. It takes a bit more time to develop than with simple methods, but it simplifies a lot the rest of the programs that build upon the class.
