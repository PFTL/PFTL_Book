\chapter{Review of Basic Operations with Python}\label{review-of-basic-operations-withpython}

\section{Chapter Objectives}\label{chapterobjectives}

The objective of \textbf{Python in the Lab} is to bring a developer from
knowing the basics of programming to be able to develop software for
controlling a complex setup. However, not all programmers have the same
background and it is important to establish a common ground from which
to start.

This chapter will quickly review how to start Python and how to interact
with it directly from the command line. It will also review some common
data structures such as lists and dictionaries. It will quickly go
through for loops and conditionals. If you are already familiar with
these concepts, you can safely skip this chapter.

\section{The Interpreter}\label{theinterpreter}
Python can be started from the command line by typing python and
pressing Enter. This should start the Python interpreter and you should
see something like this:

\begin{minted}{pycon}
Python 3.6.3 (default, Oct  3 2017, 21:45:48)
[GCC 7.2.0] on linux
Type "help", "copyright", "credits" or "license" for more information.
>>>
\end{minted}

You can type whatever you like into the interpreter. If it makes sense
to Python, it will give you an appropriate answer. For example, you
can do:

\begin{minted}{pycon}
>>> 2+3
5
\end{minted}

The first line is what you type (therefore it has the
\mintinline{pycon}{<<<} at the beginning),
while the second line is the output Python gives you (in this case, as
expected, \mintinline{pycon}{5}). You can go ahead and play with different
operations. You can multiply, subtract, divide, etc. Power of numbers
can be achieved using \mintinline{pycon}{**}, for example, \mintinline{pycon}{2**.5} means the
square root of 2.

\section{Lists}\label{lists}
From now on, the \mintinline{pycon}{<<<}
will be suppressed in order to make it easier for you to copy the code.
Python can also be used to achieve much more complicated tasks and with
many different types of variables. For example, you can have a list and
iterate over every element of it:

\begin{minted}{python}
a = [1, 2, 3, 4]
for i in range(4):
    print(a[i])
\end{minted}

As you can see, \mintinline{pycon}{a} is a list with 4 elements. You make a
\mintinline{pycon}{for} loop over the four elements and you print them to screen.
You should see an output like this:

\begin{minted}{pycon}
1
2
3
4
\end{minted}

Of course, you can argue that this is not handy if you don't know
beforehand how many elements your list has. We can improve the code by
doing it like this:

\begin{minted}{python}
for i in range(len(a)):
    print(a[i])
\end{minted}

There is an important point to note, especially for those who come from
a Matlab kind of background. If you print the variable i inside the
loop, you'll notice it starts in 0 and goes all the way to N-1. It means
that the first element in a list is accessed by the index 0. Lists have
another interesting behavior. The elements in them do not need to be of
the same type. It is completely valid to do this:

\begin{minted}{python}
a = [1, 'a', 1.1]
for i in range(len(a)):
    print(type(a[i]))
### Output ###
<class 'int'>
<class 'str'>
<class 'float'>
\end{minted}

You can have lists with lists in them and many other combinations.

\exercise{Make a list in which each element is a list. Nesting two \mintinline{pycon}{for}
loops, display all the elements of all the lists.}

Lists can also be iterated over with a much simpler syntax, without the need of the index.

\begin{minted}{python}
a = [1, 'a', 1.1]
for element in a:
    print(element)
\end{minted}

There is also a very \emph{pythonic} way of declaring lists with a very
concise syntax:

\begin{minted}{python}
a = [i for i in range(100)]
\end{minted}

This will generate a list of all the numbers from 0 to 99. You can also
calculate all the squares of those numbers with a small modification:

\begin{minted}{python}
a = [i**2 for i in range(100)]
\end{minted}

And you can make it even more complex, for example, if you want to get
only the even numbers you can type:

\begin{minted}{python}
a = [i for i in range(100) if i%2==0]
\end{minted}

\exercise{Given a list like: \mintinline{pycon}{b = [1, 2, 'a', 3, 4, 'b', 5, 'c', 'd']}, create
another list with only the elements of type string.}

Lists are a fundamental Python structure and it is important to keep
them in mind in order to follow the syntax of some programs without
getting lost.

\section{Dictionaries}\label{dictionaries}
Dictionaries are one of the most useful data structures of Python. They
are somehow like lists, but instead of accessing them via a numerical
index they are accessed via a string identifier. For example, you can
generate a dictionary and access its values by doing:

\begin{minted}{python}
a = {'first': 1, 'second': 2}
a['first']
\end{minted}

Dictionaries, like lists, can store different types of variables in them.
Pay attention to the definition and call: lists are defined using square
brackets \mintinline{pycon}{[} \mintinline{pycon}{]}, while dictionaries are defined with
curly brackets \mintinline{pycon}{{}}. However, for accessing an
element the square brackets are used. It is possible therefore to do:

\begin{minted}{python}
b = [1, 2, 3, '4', 5.1]
a = {'first': 1, 'second': b}
a['second']
\end{minted}

The first notable advantage of using dictionaries is that it makes much
clearer what data you are storing. You are giving a title to a specific
value. If you want to calculate the area of a triangle:

\begin{minted}{python}
t = {'base': 2, 'height': 1}
area = t['base']*t['height']/2
\end{minted}

And you immediately see that even if you don't have the definition of
\mintinline{pycon}{t}, it is very clear what you are doing. It is clearer than the
following code:

\begin{minted}{python}
area = t[0]*t[1]/2
\end{minted}

In the case of a triangle, it doesn't really matter which element is the
base and which one is the height. However, for more complex
applications, altering the order can have very serious consequences. In
the same fashion than with lists, it is possible to access every element
within a for loop:

\begin{minted}{python}
for key in a:
    print(key)
    print(a[key])
\end{minted}

Now the key has a value that can be printed and used. We can also check
if a specific key is present in the dictionary:

\begin{minted}{python}
if 'first' in a:
    print('First is in a')
\end{minted}

If you want to update several values of a dictionary, but not to replace
the dictionary itself, you can use the command \mintinline{pycon}{update}:

\begin{minted}{python}
a = {'first': 1, 'second': 2, 'third': 3}
new_values = {'first': 5, 'second': 6, 'fourth': 4}
a.update(new_values)
a['first']
a['third']
a['fourth']
\end{minted}

If you pay attention you will see that not only the already existent
values were updated, but a new one was created.

\exercise{Given two dictionaries, 
\mint{python}|a = {'first': 1, 'second': 2, 'third': 3}| 
and
\mint{python}|b = {'fourth': 4, 'fifth': 5}|
merge the second into the first one.}

Of course, it is also possible to delete an element from a dictionary:

\begin{minted}{python}
a = {'first': 1, 'second': 2, 'third': 3}
del a['first']
print(a['first'])
\end{minted}

You'll see an error letting you know that the key \mintinline{pycon}{first} is not
in the dictionary. So far we have always used \mintinline{pycon}{strings} for the
keys of the dictionary, but nothing prevents you from using numbers. The
following lines are perfectly valid:

\begin{minted}{python}
a = {1: 2, 2:4, 3: 9}
b = {0.1: 2, 'a': 3, 1:1}
\end{minted}

This, on one hand, makes dictionaries very versatile, on the other, it
may make the code slightly more confusing. For example \mintinline{pycon}{a[1]}
may be referring to either the second element of a list or the element
of a dictionary with key \mintinline{pycon}{1}. At this point, you may wonder why
you would use lists if dictionaries give you even more functionality.
The short answer is memory usage; the code below will output the memory
being used by a dictionary and by a list with the same information in
them. The first line of the code is just importing the function we need
for calculating the size of a variable.

\begin{minted}{python}
from sys import getsizeof

a = [i for i in range(100)]
b = {i:i for i in range(100)}

print(getsizeof(a))
print(getsizeof(b))
\end{minted}

You should see that the size of \mintinline{pycon}{a} is \mintinline{pycon}{912\ bytes} while
the size of the dictionary \mintinline{pycon}{b} is \mintinline{pycon}{4704\ bytes}. Even if
you consider that the dictionary is storing not only the value but also
the key, the ratio of memory usage of a dictionary to a list is more
than twice.

\exercise{Write a simple for loop that prints the ratio of the memory usage of a
list and of a dictionary as a function of the length of each.}
