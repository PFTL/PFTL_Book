\chapter{Run an experiment}\label{run-anexperiment}

\section{Objectives}
In this chapter, you are going to learn how to run a complete measurement. You have developed all the code you need to acquire data, but you need
to put everything together and make it work. For example, we are not saving The data yet. This chapter is also the opportunity to wrap up all what you have achieved so far and to be sure that all the classes, packages and configuration files are where
they are supposed to be.

Once you finish this chapter, you will have working knowledge on how to start controlling your own experiments in the lab. Of course, you will
need to adapt the concepts explained in the book to the devices and settings that you have, but the foundations are strong enough to build
reliable and expandable software.

\section{Introduction}
The contents of this chapter are going to become ubiquitous in almost all your lab experiments. Most likely a big part of your job is to either repeat measurements in order to improve your statistics or to change some parameters in order to find correlations in your data. A smart way of doing both tasks is through a computer with the appropriate devices hooked to it and with the appropriate software to control them.

Up to now, we have been developing a package for controlling devices. Now it is time to use it. It is a good idea to separate the files that build up the package from the ones that show a specific way of using the program. That is why we have created a separate folder for \emph{Examples}. But this is not enough, when you run the experiment and save the data, for example, you will need to do it in a different location. This will prevent polluting the base code and will allow you to separate the settings from different users working on the same computer.

Remember that in order to be able to reproduce results of your experiments, it is important to keep track of the parameters used to
perform a measurement, also known as metadata. Through this chapter, you are going to learn some ideas on how to use {YAML} files to store this information together with the data of a measurement. Moreover, you will learn how you can extend the Experiment model without writing new methods.

\section{The Layout of the Project}\label{the-layout-of-theproject}
In the previous chapters, we have developed a lot of code, and it may have ended up in the wrong location. It is important, therefore, to recap  how the structure of the package looks like:

\begin{minted}{text}
ROOT
├── Examples
│   └── Config
│       └── experiment.yml
└── PythonForTheLab
    ├── Controller
    │   ├── __init__.py
    │   └── simple_daq.py
    ├── __init__.py
    ├── Model
    │   ├── daq
    │   │   ├── analog_daq.py
    │   │   ├── base.py
    │   │   ├── dummy_daq.py
    │   │   └── __init__.py
    │   ├── experiment
    │   │   ├── __init__.py
    │   │   └── IV_measurement.py
    │   └── __init__.py
    └── View
\end{minted}

It is strongly recommended that you copy the structure of the project. This will allow you to follow the code without having to adapt all the import statements. Pay attention to the fact that every folder has its own \emph{\_\_init\_\_.py} file, which we have discussed earlier, and in some cases we have used it to define useful components, such as \texttt{Q\_}. If you see something that you can't understand, you can always refer to the repository on Github\footnote{https://github.com/PFTL/SimpleDaq}.

There is a topic that normally generates headaches with newcomers to Python: how importing works. Imagine that we would like to use the \texttt{DummyDaq} model. We can be tempted to do the following: 

\begin{minted}{pycon}
>>> from Model.daq.dummy_daq import DummyDaq
\end{minted}

which will work only if we have started the Python interpreter inside the PythonForTheLab folder. When you import, Python will try to find what you want to load in a very specific order and in very special folders. If we are inside the Examples folder, for example, how can Python know that it is supposed to go one level up and into PythonForTheLab to find the \emph{Model} module? 

There are different ways of solving this problem. The easiest is to always start Python in the root folder. In that case, we will be able to import whatever module we are interested like this:

\begin{minted}{pycon}
>>> from PythonForTheLab import Q_
>>> from PythonForTheLab.Model.daq.dummy_daq import DummyDaq
\end{minted}

This also means that if you have a script in the Examples folder, you would run it like this:

\begin{minted}{bash}
python Examples/my_script.py
\end{minted}

This works, but it is not very handy. At some point you would like to run a script which is in a different place in your computer, or you would like to be able to build on the PythonForTheLab package. This is all possible, but it is very dependent on the operating system which you are using. You can refer to section \ref{section:importing-python}, in which we discussed how to add folders to the Python path. If you are using an IDE such as Pycharm or Visual Studio Code, these steps are taken care for you, provided that you trigger your program from within the editor. 

A more elegant solution will be discussed at the end of the book, when we develop a \textttt{setup.py} script that will allow us to install our package in the environment in which we work and not worrying about paths, etc. After you have added the folder the {PATH} in your computer, you can start working in any folder that you wish. The files that we are going to create from now on in this chapter are not going to be part of the main package, the recommendation is to work inside the \emph{Examples} folder, but you are free to choose. 

\section{Running an Experiment}\label{running-anexperiment}
We have the configuration file defined from the previous chapter. Therefore, we can start using the experiment model right away. We can start by creating a file in the \emph{Examples} folder, let's call it \textbf{start\_cli.py}, and add the following:

\begin{minted}{python}
from PythonForTheLab.Model import Experiment

e = Experiment()
e.load_config('Config/experiment.yml')
e.load_daq()
\end{minted}

Remember to change the path to the {YAML} file with the configuration for your experiment. The advantage of this approach is that you could have several configuration files for different experiments and you load the one you want. For example, you may want to do a coarse scan just to check everything is well connected, or you may want to decrease the delay, etc. 

\exercise{You already know how to trigger a scan with the default parameters specified in the configuration file. Perform a measurement and print the data to screen.}

If you use the experiment model as is, and you trigger the \mintinline{python}{do_scan} method, you will notice several things. The first is that you don't know if the program is running or not. When you are facing this situation, the easiest solution is to use some \mintinline{python}{print} statements in order to know where you are at. For example, you can add 
\begin{minted}{python}
print('Starting scan...')
\end{minted} 
Right before the scan starts, and you can do the same right after it finished. This, at least, will allow you to see something happening on the screen. You could also think about using print statements in the experiment model, so you can see what values are being passed to the DAQ device. This all seems reasonable at the beginning, but be careful. Printing to screen is a quick way of distracting users with too much input. Imagine you are scanning 1 million points and you get a message for each one of them. Moreover, we are going to improve on this later in this chapter. 

The other thing that perhaps you have realized is that if you want to stop the scan, you have to interrupt the execution of the program, which leads to unexpected results. For example, you are going to lose the acquired data, and the DAQ will be left in a state which may not be safe. Finally, you probably realized that the program blocks the execution of any other code while the scan is running. If you want to plot the progress, you will have to wait until it is finished. We are going to solve all of this. However, there is something important to point out first.

Once you instantiate the Experiment class and load the config file, you will have a dictionary located at \mintinline{python}{e.properties} with all the information contained in \textbf{experiment.yml}. Those parameters are used by the \mintinline{python}{do_scan} method, but nothing prevents you from changing them before doing it. For example, you can add the following to the code:

\begin{minted}{python}
e.properties['Experiment']['Scan']['start'] = '0.5V'
e.properties['Experiment']['Scan']['step'] = '500mV'
e.do_scan()
\end{minted}

When the scan is performed, the parameters that are going to be used are the ones just defined, and not the ones in the {YAML} file. This is very handy because you could even wrap the code in order to study the dependence of the results with some parameters which perhaps you didn't think of before. For example, imagine that you want to see if the {IV} curve depends on the step size or the delay between points. You could do the following:

\begin{minted}{python}
for delay in range(10, 100):
    d = delay*Q_('ms')
    e.properties['Experiment']['Scan']['delay'] = d
    e.do_scan()
\end{minted}

You can see that by having a model class for the experiment, the biggest work was already resolved. There are some details that you will have to address, for example, how to save the data, but they are normally simple to resolve. Also, you have to remember that the scan is happening with parameters not defined in the {YAML} file. In order to maintain the idea of doing reproducible research, you should always save the parameters you used as part of the results.

\section{Plot results and save data}\label{plot-results-and-savedata}
We know how to perform one or several measurements, but now we need to do something with the data we have just acquired. First, we are going to plot it and then we are going to save it to a text file.

For plotting we are going to use a library called \href{http://www.pyqtgraph.org/}{PyQtGraph} which was developed mainly to generate fast data visualizations. Exactly what we need when we are dealing with experiments. If you are familiar with any other plotting libraries such as matplotlib, feel more than free to adapt the code to use them. The examples in this section are based on PyQtGraph because it is the library that we are going to use while building a {GUI}. The first thing you have to do is to import the library and make a plot of the data.

\begin{minted}{python}
import pyqtgraph as pg

pg.plot(e.scan_values, e.scan_data)
\end{minted}

It is a super simple example but it works. You are using the data stored directly within the class. If you called the data differently, adapt the code so you can use it. At this point, you should see the curve of your results appearing. If you want to add labels and a title to the plot, you can do the following:

\begin{minted}{python}
PlotWidget = pg.plot(title="Plotting I vs V")
PlotWidget.setLabel('bottom', 'Channel: {}'.format(
    e.properties['Scan']['channel_out']), units="V")
    
PlotWidget.setLabel('left', 'Channel: {}'.format(
    e.properties['Scan']['channel_in']), units="V")
    
PlotWidget.plot(e.scan_values, e.scan_data)
\end{minted}

There is no much complexity at this point; once you have the data the number of things you can do is endless. You can fit a model, repeat a new scan with parameters based on your results, etc. However, saving the data is crucial if you are intending to use it in a publication or if you want to do a better analysis at a later stage. To save the data, we can do the following:

\begin{minted}{python}
from datetime import datetime

filename = "Data_scan_{:03}.dat"

header = "# Data saved by Python For The Lab\n"
header += "# Creation date: {:%Y-%m-%d %H:%M:%S}\n".format(
    datetime.now())

header += "# Number of lines: {}\n".format(len(e.scan_values))
header += "##################################\n"
\end{minted}

First, you define the \texttt{filename} in such a way that we can add numbers to it. The \texttt{\{:03\}} means that there are going to be 3 spaces for the number, therefore the filename will look like \textbf{Data\_scan\_001.dat}, \textbf{Data\_scan\_002.dat}, etc. You can try it if you do something like \mintinline{python}{print(filename.format(34))}. Later we define the header of the file. The header has to describe what is stored in the file. We have added a title, the creation date, and the number of points we have acquired. The creation date is essential for keeping a tidy record because it will allow us to match the data with whatever we have written in the lab journal. The number of lines is a bit old-fashioned; it is very useful when you are reading line by line and you need to stop right at the last line before an error occurs. Note that at the end of each line there is a \mintinline{python}{\n} in order to force a new line after each statement.

Now we have to be sure that we are not overriding a previous file with the new one. We need to check if the file exists, like this:

\begin{minted}{python}
import os
os.path.exists(filename.format(0))
\end{minted}

The command will return \mintinline{python}{True} if the file \mintinline{python}{Data_scan_000.dat} exists or \mintinline{python}{False} if not.

\exercise{Check what is the smallest number for which a file doesn't exist. \emph{Hint}: the \mintinline{python}{while} loop is your best friend in this case.}

Once we know which number is available for saving our file, you will just write the data. Assuming that the number for the file is
\mintinline{python}{i}:

\begin{minted}{python}
with open(filename.format(i), 'w') as f:
    f.write(header)
    for i in range(len(e.scan_values)):
        line = "{:4.4f}, {:4.4f}\n".format(
                e.scan_values[i], e.scan_data[i])
        f.write(line)
\end{minted}

As you can see, the header is written in just one go. Thanks to the \mintinline{python}{\n} characters that were appended to every line it is going to be correctly displayed if you open the file with any text editor. The saving of the data itself happens in a for loop, that covers every data point acquired. The format specifies that data is going to be saved with a precision of 4 decimal points. If you want to read about the possible formats, you can check \href{https://pyformat.info/}{PyFormat} or the \href{https://docs.python.org/3.6/library/stdtypes.html\#str.format}{Python Documentation}. Go ahead and open the file to see how does the final version look like.

# 01/08/19

If you are confident with the results, you can add this procedure as a method in the experiment model. You can also add a parameter to the yaml file for specifying the name of the files that you are producing. Remember, if you change the code and a new parameter is needed, you have to update the file in the \emph{Examples} folder to reflect it. If you fail to do so, you will be solving a lot of issues every time someone wants to start a new experiment.

\section{Running the scan in a nonblocking way}\label{running-the-scan-in-a-nonblockingway}

When a function takes long to execute, such as the case fo the
\mintinline{python}{do_scan}, it is said that the function is blocking the rest of
the program. Sometimes the functions take long to execute because they
are computationally expensive, sometimes they take longer to execute
because the process is slow. For instance, when you do a scan, the
majority of the time the program is waiting in a \mintinline{python}{sleep} before
changing the value and reading again. Waiting is not computationally
expensive and therefore your computer is capable of performing other
tasks at the same time.

The idea of what you want to achieve is that the long function, in this
case, \mintinline{python}{do_scan} should run in the background while the rest of
the program is able to do some other things. For example, you can plot
the results while you are acquiring them, or you wish to stop the scan
without killing the program. Python offers a very interesting package
called \mintinline{python}{threading}, which will allow you to do exactly what you
are looking for.

\section{Threads in Python}\label{threads-inpython}
You can think about threads as different lines of execution. A Thread
would be similar to opening different terminals and running Python in
each one of them. The advantage is that since you are running all your
threads in the same Python interpreter, you can easily exchange
information within them. Let's see first a simple example of a function that takes long to execute:

\begin{minted}{python}
import threading
from time import sleep

def func(steps):
    for i in range(steps):
        sleep(1)
        print('Step: {}'.format(i))

t = threading.Thread(target=func, args=(3, ))
print('Here')
t.start()
sleep(2)
print('There')
\end{minted}

If you run the code above, you should see the following output:

\begin{minted}{text}
Here
Step: 0
Step: 1
There
Step: 2
Step: 3
\end{minted}

What is happening is very interesting. If you run the function
\mintinline{python}{func} on its own, you will see that the numbers appear one by
one, and while the function doesn't end, you won't be able to do
anything else. However, when you create a thread, i.e. when you execute
\mintinline{python}{threading.Thread}, the target function is going to run in a
separate space, therefore not blocking your program. When you create a
\emph{thread}, you have to specify the function you are going to
execute, bearing in mind that it is the function itself and not the
instance, i.e., you don't append the \mintinline{python}{()}. If the function takes
arguments, you add them as a comma-separated list, even if it is only
one as in the example above.

When you create the thread, the function is not executed, it is waiting
for the \mintinline{python}{start()} signal to start running. What you see is that
even after the thread starts, the
\mintinline{python}{print('There')} statement is
being executed. First, you print
\mintinline{python}{'Here'}, you start the
\emph{thread}, you wait for two seconds, in which \mintinline{python}{func} prints
its first two steps, you print
\mintinline{python}{'There'} while the thread
continues its execution. Working with threads can get very complicated
when you need to share information, or when you need to be sure one
specific thread finished before you do something else, etc. For simple
examples, however, it is very handy and easy to understand.

You can also have several threads running at the same time, and you can
assign names to them:

Python

\begin{minted}{python}
def func(steps):
    print(threading.currentThread().getName(), 'Starting')
    for i in range(steps):
        sleep(1)
        print('Step: {}'.format(i))
    print(threading.currentThread().getName(), 'Finishing')


first_t = threading.Thread(name='First Thread', 
                        target=func, args=(3, ))
second_t = threading.Thread(name='Second Thread', 
                        target=func, args=(3, ))
print('Here')
first_t.start()
second_t.start()
sleep(2)
print('There')
\end{minted}

Go ahead and run the script to see what happens.

\section{Threads for the experiment model}\label{threads-for-the-experimentmodel}

Going back to the experiment model, you can already test what you have
learned by executing the \mintinline{python}{do_scan} method in its own thread.
Since the method doesn't take any arguments, the line should look like:

\begin{minted}{python}
t = threading.Thread(target=e.do_scan)
\end{minted}

While the scan is running in the background, you can do other things in
the main thread, such as plotting. Let's see how you can do that. You
need to refresh the plot while the acquisition is happening, and
therefore you will need to update the plot within a loop. One possible
way of doing it is like this:

\begin{minted}{python}
import threading
import pyqtgraph as pg
from time import sleep
from PythonForTheLab.Model import Experiment

e = Experiment()
e.load_config('config.yml')
e.load_daq()
t = threading.Thread(target=e.do_scan)
t.start()

PlotWidget = pg.plot(title="Plotting I vs V")
PlotWidget.setLabel('bottom', 'Port: {}'.format(
        e.properties['Scan']['port_out']), units="V")
PlotWidget.setLabel('left', 'Port: {}'.format(
        e.properties['Scan']['port_in']), units="V")
PlotWidget.plot(e.xdata_scan, e.ydata_scan)

while t.isAlive():
    PlotWidget.plot(e.xdata_scan, e.ydata_scan, clear=True)
    pg.QtGui.QApplication.processEvents()
    sleep(1)
\end{minted}

The beginning of the code is a combination of what you have been doing
for running the scan and plotting and the use of threads. The important
part is what is happening in the \mintinline{python}{while} loop. First, you check
that the thread is still running with the \mintinline{python}{isAlive()} method.
Then, within the loop, you update the plot with the data in the
experiment class. It is very important to note that even if the
\mintinline{python}{do_scan} method is running in a different thread, its data is
accessible from the main thread. There is an extra line in which you
declare a \mintinline{python}{QApplication}. For the time being, do not worry about
it. Its only purpose is to use the plot that you have already created a
few lines above.

Exchanging information with a class means that you cannot only read from
it but that you can also write to it. Therefore, a clever way of
stopping a scan is by checking the status of a property within the loop.
For example, you can add the following code to the \mintinline{python}{do_scan}
method of the experiment model:

\begin{minted}{python}
if self.stop_scan:
    break
\end{minted}

What is happening is that if the property \mintinline{python}{stop_scan} is set to
\mintinline{python}{True}, the for-loop in which you are running the scan will stop.
Remember that if you add a new property such as \mintinline{python}{stop_scan}, you
have to initialize it in the \mintinline{python}{__init__} method of the class.
It is also important to set the property to \mintinline{python}{False} every time
you start a new scan. If you fail to do that, after you stop a scan you
won't be able to run another one. The \mintinline{python}{do_scan} method will look
like this:

\begin{minted}{python}
def do_scan(self):
    self.stop_scan = False
    [...]
    for value in scan:
        if self.stop_scan:
            break
    [...]
\end{minted}

The \mintinline{python}{[...]} means that there is code that is not being
displayed, for brevity.

\exercise{Run the example with the threads again, but this time change the value
of \mintinline{python}{e.stop_scan} to \mintinline{python}{True} after a certain number of plot
updates.}

\exercise{Use the input from the keyboard to stop the scan}

Finally, it is important to point out that sometimes you actually need
to wait for a thread to finish. Imagine that you want to perform several
scans, you don't want to do them at the same time, but one after the
other. To wait for a thread to finish you can use the method
\mintinline{python}{join()}, like this:

\begin{minted}{python}
t = threading.Thread(target=e.do_scan)
t.start()
print('Thread Running')
t.join()
print('Thread Stopped')
\end{minted}

In this example, \mintinline{python}{Thread Stopped} will appear only after the
Thread has finished. Moreover, when you start working with this design
pattern in mind, you will realize that there is nothing that forbids the
user from triggering two scans at the same time, in two different
threads. This can give a lot of headaches when you start developing
complex applications. Sometimes, the errors that appear are going to be
very hard to debug. It is very hard to transmit all the different things
that can happen once a different person starts using your software.

\exercise{Add an extra property to the experiment model called
\mintinline{python}{scan_is_running}. Set it to \mintinline{python}{True} when the scan runs
and \mintinline{python}{False} when the scan stops. Use it to avoid two scans to run
simultaneously.}

Threads work also in Jupyter notebooks. If you start a thread in a cell,
you will see that it immediately releases the interpreter. You can move
to a different thread and keep working, updating a plot, etc. Whenever
you want to stop the scan, you can do it as before, changing the value
of the \mintinline{python}{stop_scan} property.

\section{Word of Caution with Threads}\label{word-of-caution-withthreads}

Threads are a very complex topic in any programming language, and even
if the examples that you have seen up to now seem straightforward, you
need to be cautious about how you implement threads in larger programs.
Today computers with multi-core processors are ubiquitous. However, when
you run a Python program it is going to run only within one of the cores
of the computer. Moreover, each core can run only one thread at a time.

When you run multiple threads, the processor very quickly switches from
one thread to another. The first thread runs for a while, then it stops,
another thread runs for a while, stops, etc. Switching from a thread to
another is not free for the processor and even if the process is not
doing anything, the processor has no way of knowing it and will switch
to it anyways. While you have processes like the ones you have developed
up to now, which are not computationally expensive, you won't notice
performance issues. However, if you start dealing with image analysis,
data processing, etc. you will have to find better alternatives.

If you want to test the limits of threading in Python, you can generate
large random arrays. Run the process on different threads and check the
status of your computer. You will notice that not all the cores
available are in use, but one of them is going to be at 100\%. Moreover,
you can check what takes longer if generating 4 random arrays in 4
different threads or one after the other.

You shouldn't confuse the idea of multithreading with parallelizing
code. When you parallelize code you are able to use all the cores of
your computer at the same time. Threading just allows you to run code at
the same time on the same processor. The subject is vast and exceeds the
objectives of this book. However, you need to be aware of the
limitations of the techniques that you use and the degree of
applicability that they have. If you want to develop code that runs in
parallel, you can look at libraries such as \emph{mpi} or
\emph{multiprocessing}.

\section{Conclusions}\label{conclusions}
This is a chapter where you can see how all your previous work plays out
nicely together. Developing drivers and models may seem a lot of
unnecessary work, but when you see everything coming together you start
to realize that it really pays off being organized and systematic with
the code. If you stop reading the book in this chapter, you would have
already acquired a lot of knowledge that can improve your work in the
lab. You have learned the {MVC} pattern, you have developed models, and
worked with threads. You have also seen how to store metadata and use it
to run an experiment again.

The most important message that you should take with you after these few
chapters is that the code you write today should be useful tomorrow.
You have started with simple tasks, such as communicating with a serial
device, and some chapters later, the same code is being used for
performing experiments. If you think about it, the base driver was never
updated since you first developed it, it just changed folders. This is,
in principle, what you expect to happen with all the code you develop.
Each class, each function, can be a building block for something else.
You will see how this materializes in the next two chapters when you
develop a Graphical User Interface ({GUI}).

When you work in the lab, you will notice that every device is
different, has different requisites, and a different behavior. It is
very important that you understand, first of all, what do you want to do
with the devices that you have available. Devices for the lab are not
home appliances and may have options which you should understand by
reading the manual. Perhaps you need external triggers, an amplifier, a
filter, etc. Once you understand what experiment you are actually
performing, then you can start programming, but not the other
way around.
