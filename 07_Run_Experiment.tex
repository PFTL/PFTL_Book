\chapter{Running an Experiment}\label{ch:run-experiment}

\section{Introduction}\label{sec:run-experiment-introduction}
In the previous chapter, you've seen how to develop an experiment model and how to use it from a simple script. In this chapter, you're going to explore how to bring it to the next level. So far, you can only run the experiment with the parameters specified in the config file, but you're not limited to them. Moreover, once the scan starts, there's no way of stopping it unless you completely stop the program. That leads to missing data and is not particularly helpful.

Following the \emph{Onion Principle}, it's now time to bring the experiment model to the next level.

\section{Running an Experiment}\label{sec:running-experiment}
In the previous chapter, you've seen how to run an experiment from a script. Each of the steps needed is triggered one after the other. You called that script \textbf{run\_experiment.py} and it contained the following code (which skips some less relevant lines):

\begin{minted}{python}
from PythonForTheLab.Model import Experiment

experiment = Experiment('experiment.yml')
experiment.load_config()

experiment.load_daq()

experiment.do_scan()
experiment.save_data()
experiment.finalize()
\end{minted}

It's relatively straightforward, but this is not the only thing that you can do with the experiment. You can access the attributes of the experiment class while you're performing a measurement. For example, you could show the data once the scan finishes:

\begin{minted}{python}
experiment.do_scan()
print(experiment.scan_data)
\end{minted}

However, you could also change the parameters of the scan after loading the config file. Sharing information with classes in Python always occurs in one of two ways: you can read from them, but you can also write to them. Let's see what happens when you change the \py{config} attribute of the experiment before triggering the scan. You can do something like the following:

%! Suppress = Ellipsis
\begin{minted}{python}
[...]

experiment.config['Scan']['num_steps'] = 5
experiment.do_scan()
experiment.save_data()
experiment.config['Scan']['num_steps'] = 10
experiment.do_scan()
experiment.save_data()
experiment.finalize()
\end{minted}

You can change any of the values stored in the \py{config} of the experiment at any time. You can change them from the running script, for example, but you can also change them from the Python interpreter. For people familiar with Jupyter notebooks, for instance, the experiments can be run directly from within them. There are some extra benefits of using notebooks, such as the possibility of documenting your work. You can learn more about them in the Advanced Python for the Lab book.

\tipsInfo{BPython}{Running Python from the interpreter usually is cumbersome. There's another version of the interpreter called \textbf{bpython} which has auto-complete functions and allows you to edit code before running it. It's not recommended for anyone to run experiments from the interpreter, but sometimes it's the handiest solution to change values when desired.}

In the code above, you've changed the value of the number of steps before triggering a scan. First, you've set it to 5 steps, and then to 10 steps. You can check the metadata and see that this information was saved. If you open the data file, you would also see a different number of data points. This flow suggests some exciting possibilities. For example, if you would like to see if there's any form of hysteresis on the measurements, you could vary the delay between data points.

\questionInfo{Exercise}{Write a for-loop that changes the delay between data points after each scan. Remember to save the data so you can explore the differences.}

When you start developing complex logic in the running script, it's fair to wonder if it's better to put all that logic into the experiment class. As always, when you perform a task only once, knowing it will be a one-off task, then it's quicker to run it from a script. If, however, that one-off task becomes an essential part of your experiments, or you believe that others could benefit from performing the same measurement, then you should implement the code in the class. As always, common sense is the best ally for a scientific developer.

One of the important things to note is that when you work with units, you leave the values as a string, and only transform them into real quantities when you need them. If you explore the \py{.do_scan()} method, you'll see that you start by running the parameters through the unit registry. This means that if you want to change the values of, for instance, the start, stop, or delay, you only need to change a string:

\begin{minted}{python}
    experiment.config['Scan']['start'] = '2.2V'
    experiment.config['Scan']['delay'] = '200ms'
\end{minted}

\section{Plotting Scan Data}\label{sec:basic-plotting}
You've learned how to do several things with the device, but you're not doing anything with the data after you save it. It would be a great addition to your workflow to be able to see what you're doing through a visual plot. For this step, you're going to use a library called PyQtGraph, which was developed mainly to generate fast data visualizations. This is exactly what you would need when you're dealing with experiments. Usually, those a bit more familiar with Python would become acquainted with matplotlib, which is an excellent tool for paper-ready plots, though a bit slow for real-time visualization. In any case, at this stage, you don't care about speed, so if you're more confident with another tool, then feel free to adapt the code to use it. In any case, you'll see PyQtGraph again later, when you build the user interface.

Making a simple plot with PyQtGraph is very easy. You can add the following after you perform the scan, in the \textbf{run\_experiment.py} script file:

%! Suppress = Ellipsis
\begin{minted}{python}
import pyqtgraph as pg

[...]

pg.plot(experiment.scan_range, experiment.scan_data)
\end{minted}

The caveat with this solution is that you must run the program slightly differently:

\begin{minted}{bash}
python -i run_experiment.py
\end{minted}

You must append the \py{-i} or the plot will close right after it appears on screen. It's simple, but it works. PyQtGraph also gives you some mouse interactions out of the box. You can drag the image, zoom in and out with the mouse wheel, and even transform the scale of the axes by right-clicking. You can make the plot more aesthetically pleasing by adding labels and a title:

\begin{minted}{python}
PlotWidget = pg.plot(title="Plotting I vs V")
PlotWidget.setLabel('bottom', f"Channel: {experiment.config['Scan']['channel_out']}", units = "V")
PlotWidget.setLabel('left', f"Channel: {experiment.config['Scan']['channel_in']}", units = "V")
PlotWidget.plot(experiment.scan_range, experiment.scan_data)
\end{minted}

Plotting with PyQtGraph is relatively simple. However, you can only plot data after the scan finishes. The \py{.do_scan()} method takes a relatively long time to complete. When a method or function works in this way, they're called \textbf{blocking} functions. The program can't continue until they finish. If you would like to monitor the progress of the experiment while it's running, then you need to find a way of running the scan in a non-blocking manner.

\section{Running the Scan in a Non-Blocking Manner}\label{sec:nonblocking}
Methods or functions like \py{.do_scan()} take a long time to run. Imagine that you would like to acquire a movie for one hour. In that case, the function would take one hour to complete. If you were to run a very complex simulation or data analysis process, your functions could also take a very long time to complete. However, both scenarios are different from each other. Sometimes functions take long to execute because they are \textbf{computationally expensive}. They need many cycles of the processor to finish. Other times, they take longer to execute because the process is slow. When you do a scan, the program waits in a \py{python}{sleep} or waits for the device to read a signal. Waiting is not computationally expensive, which means that your computer can perform other tasks at the same time if you know how to make it do so.

Python has different options for achieving this. Still, the one that gives the best results, not only in terms of flexibility but also in terms of simplicity of implementation, is the \emph{multithreading} module\footnote{Another library which is gaining popularity is AsyncIO. It is, however, harder to efficiently implement it for the purposes of this book.}.

\subsection{Threading in Python}\label{subsec:multithreading}
All threads have two ends. A computer program always starts going through a thread in the same direction until it reaches its end. Sometimes, along the thread, a task \textbf{blocks} the progress, and the program halts there for some time. What Python allows you to do is to start other threads at any point in time. The more threads you have, the more entangled the program will be. Python, however, does not run the threads at the same time. There's a tool called the \textbf{Global Interpreter Lock (or GIL)} that prevents two things from happening simultaneously.

What Python does is run small pieces of each thread one after the other. Someone once compared it to a super-efficient secretary, who pushes jobs according to who has free time and who has tasks to perform. It means that two computations won't happen at exactly the same time, but if one thread is waiting on a \py{sleep}, for instance, Python can go ahead and let another thread run. Whenever you start Python, you're starting a thread, also called the \emph{main thread}, which lives through your program. From that thread you can start a second one, as you do in the code below:

\begin{minted}{python}
import threading
from time import sleep

def func(steps):
    for i in range(steps):
        sleep(1)
        print('Step: {}'.format(i))

t = threading.Thread(target=func, args=(3, ))
print('Here')
t.start()
sleep(2)
print('There')
\end{minted}

If you run it, you get the following output:

\begin{minted}{text}
Here
Step: 0
Step: 1
There
Step: 2
\end{minted}

It's essential for you to understand what's happening in the code above, so let's dissect it step by step. If you run the function \py{func} on its own, you see that the numbers appear one by one. You'll also notice that while the function is running, nothing else happens. However, in the code above, you can see a \py{'There'} printed in between the output of the function. It means that you managed to trigger the function, and it didn't stop the execution of the rest of the program.

When you used \py{threading.Thread}, you were creating a new child thread. The \py{target} is whatever function you want to run on that thread. You can also pass arguments, as you did in the code block above, using \py{args=(3, )}. Once you create the thread, you have to start it by calling \py{t.start()}. Note that the \py{target} is \py{func} and not \py{func()}. If you would use \py{func()}, you would be using the result of the function, and not the function itself. Therefore, there wouldn't be much to gain from the thread.

\criticalInfo{Functions or Function Calls}{It's very important for you to be able to distinguish between the proper function and its output. To run a function on a separate thread, you use \py{target=func}. If you call the function instead, you would be sending its \emph{output} to a thread, which is not useful. If things don't work, check whether there's an extra pair of parentheses \py{()} after \py{func}.}

Another common pitfall is forgetting the \py{start()}. When you create a new thread, the function is not called, but it waits until the \py{start()} signal is triggered. That is why you first see \py{'Here'} being printed, then some steps that follow. With such a simple example, it's always a good idea to change the parameters to see how they affect the output printed to screen, and what happens first and second.

You can also have several threads running the same function at the same time. In the example below you can see how easy it is:

\begin{minted}{python}
first_t = threading.Thread(target=func, args=(3, ))
second_t = threading.Thread(target=func, args=(3, ))
print('Here')
first_t.start()
second_t.start()
sleep(1)
first_t.join()
second_t.join()
print('There')
\end{minted}

Starting two threads is as simple as starting one. You've added one extra detail to your code,using \py{join()} to wait for the threads to complete. Now you can see that the program prints \py{'There'} after the function finishes. Using \py{join()} is usually a good idea in order to not finish the program and leave some orphaned threads.

\infoInfo{Threads don't run in parallel}{The fact that things can happen more or less at the same time does not mean the code is running in parallel. Threads have been around for a very, very long time. Even in single-core computers, users were able to switch from one program to another. You could get e-mails while browsing the internet. It was the operating system taking care of executing bits of each program to keep them all up to date. Parallelization means running computations at the same time. In multi-core processors like the ones available in most computers and cell phones today, you can split tasks and run them at the same time in different cores. For truly running on different cores, Python offers the \emph{multiprocessing} package. However, multiprocessing is challenging. The Advanced Python for the Lab book covers in detail how to leverage it for more extensive and more data-consuming experiments.}

\section{Threading for the Experiment Model}\label{sec:threads-experiment-model}
You've seen how to run a straightforward function on a different thread, but you can also see how more complex functions do not present a challenge. Going back to the experiment model, you can already test what you've learned by running the \py{do_scan} method on its thread. Since the method doesn't take any arguments, the code would simply look like this:

\begin{minted}{python}
    t = threading.Thread(target=experiment.do_scan)
    t.start()
\end{minted}

While the scan is running in its thread, you can do other things in the main thread, such as plotting. You need to refresh the plot while the acquisition is happening, and so you need to update the plot within a loop. Combining what you've seen so far, you can update the \textbf{run\_experiment.py} file:

\begin{minted}{python}
from time import sleep
import pyqtgraph as pg
import threading
from PythonForTheLab.Model.experiment import Experiment


experiment = Experiment('experiment.yml')
experiment.load_config()
experiment.load_daq()
t = threading.Thread(target=experiment.do_scan)
t.start()

PlotWidget = pg.plot(title="Plotting I vs V")
PlotWidget.setLabel('bottom', f"Channel: {experiment.config['Scan']['channel_out']}", units = "V")
PlotWidget.setLabel('left', f"Channel: {experiment.config['Scan']['channel_in']}", units = "V")

while t.is_alive():
    PlotWidget.plot(experiment.scan_range, experiment.scan_data, clear=True)
    pg.QtGui.QApplication.processEvents()
    sleep(.1)

experiment.finalize()
\end{minted}

The beginning of the code is the same. The only difference is that you trigger the scan inside its thread. After starting it, you create a plot exactly in the same way as you did before. The important part is the \py{while} loop. First, you check whether the thread is still running with the \py{is_alive()} method. If the thread is not running, it means the scan has finished. Then, within the loop, you update the plot with the data available in the experiment class. It's crucial to note that even if the \py{do_scan} method is running in a different thread, its data is accessible from the main thread. The extra line with the \py{QApplication} is necessary to make things work, but it's not important, and since you're not going to follow this approach in this book, you won't see it discussed further.

Exchanging information between threads is a topic that needs to be handled with care. Right now, there are two threads: the main thread, in which you define the experiment and the plot, and the child thread, in which the scan is happening. However, for a scan to happen, the thread needs to have a copy of the experiment itself. It turns out that it's not just a copy, but the experiment is the same object on both threads. That is why, if the child thread modifies the values of \py{scan_data}, for example, the main thread can see them. However, data is not the only thing that is shared. The experiment communicates with the {PFTL DAQ} device. This communication is open to both the main and the child threads. Nothing prevents you from triggering two scans at the same time:

\begin{minted}{python}
    scan1 = threading.Thread(target=experiment.do_scan)
    scan2 = threading.Thread(target=experiment.do_scan)
    scan1.start()
    scan2.start()
\end{minted}

If you do something like this, you see the inherent problem of working with threads carelessly. Each scan has a for-loop, in which the output voltage changes to a given value, and then a voltage is read. In the best-case scenario, when one thread sets the value, the other changes it, and the voltage read corresponds to the second one. In the worst-case scenario, the information transmitted to and from the device gets split. This means that the bytes transmitted to and from the device can end up intercalated. It corrupts the data and can lead to crashes or the device going beyond the specified range of voltages.

You won't see here the details of all the possible solutions to prevent these problems from appearing. The threading package has many tools to make your programs \emph{thread-safe}. However, you also have to find a balance between how complex you want to make your solution and how much you can trust that you won't make these kinds of mistakes. Here, you use a straightforward approach that prevents you from triggering a second scan if one is already running. That is a likely scenario if you don't realize a scan is taking place. The solution is relatively easy: when the scan is running, you set a variable to \py{True}. Every time you want to start a new scan, you check the variable and prevent the program from going further if a scan is already happening. You must edit the \py{Experiment} class:

%! Suppress = Ellipsis
\begin{minted}{python}
class Experiment:
    def __init__(self, config_file):
        self.is_running = False  # Variable to check if the scan is running
    [...]

    def do_scan(self):
        if self.is_running:
            print('Scan already running')
            return
        self.is_running = True
        [...]
        for volt in self.scan_range:
            [...]
        self.is_running = False
\end{minted}

All the code that hadn't changed was removed to truncate the output. It's very important to define the \py{self.is_running} attribute in the \py{__init__} because if you don't, the program crashes the first time you try to run a scan. Then, you check if the scan is already running. If it is, you print a message and stop the execution by using a \py{return}. This pattern is convenient to avoid using a very long if-else block. Then you switch the attribute right before starting the loop and back to false when it finishes. It should be enough to prevent two threads from running a scan at the same time. You can go ahead and re-run the script to see that this time you get a nice message warning you that a scan is already taking place.

\criticalInfo{Thread safety}{Those experienced with threads may dislike the solution above, and they are right to do so. There's a chance that both threads will check if the scan is running precisely one after the other, and then both see that it's not. Then, both threads set the safety variable to \py{True}, and you face the same issues as before. This situation can happen if you trigger two threads to do the scan one right after the other. In practical terms, however, you trigger the scan by clicking on a button, or by typing in the Python interpreter, and this is never quicker than checking whether a variable is true or not. However, as programs grow in complexity, these concerns can become real issues\footnote{We have written an extensive tutorial on threading on our website. Feel free to check it out to learn more.}}

There's only one extra feature that your \py{do_scan} method is missing: the ability to stop whenever you want. You've seen that the main thread can read data stored in the experiment class even if a child thread generates this data; but remember also that the child thread can see the changes to the attributes of the experiment class. Therefore, you can use an attribute, similar to the \py{is_running}, that signals that you want to stop the scan. You can modify the \py{do_scan} again:

%! Suppress = Ellipsis
\begin{minted}{python}
    def do_scan(self):
        [...]
        self.keep_running = True
        for volt in self.scan_range:
            if not self.keep_running:
                break
\end{minted}
When you start the scan, \py{keep_running} is set to \py{True} because you want to keep running the scan. Then, in every iteration, you check whether this variable changed or not. If it's \py{False}, then the loop would stop. You can see how this would work in the \textbf{run\_experiment.py} script:

%! Suppress = Ellipsis
\begin{minted}{python}
[...]
scan1 = threading.Thread(target=experiment.do_scan)
scan1.start()
sleep(2)
experiment.keep_running = False
print('Experiment finished')
experiment.finalize()
\end{minted}

The example is relatively straightforward. You start the scan, wait for two seconds, and then you stop it. It's not a particularly useful situation, but it's crucial for cases when you want to be in control and not lose data or damage the equipment.

\questionInfo{Exercise}{Use the input from the keyboard to stop the scan. The best approach is to wait while the thread is alive, such as what you did for plotting. Then, you can use a try/except block, using the KeyboardInterrupt exception.}

\section{Improving the Experiment Class}\label{sec:improving-experiment}
Throughout the book, you've always come back to the \emph{Onion Principle}. Every time you find something that you believe can be useful in the future, you shouldn't leave it as an example in a Python script, but you should instead try to implement it in a robust way. This is exactly what you did with threads. You've seen that you can run the scan without blocking the program, but to achieve it, you have to remember how to work with threads. A better idea would be to implement the threads directly in the Experiment class.

In this case, you can write a new method that takes care of starting the scan in a separate thread and another method just for stopping. Having specific methods is a perfect way to not have to remember which attributes do what. Let's create the methods for starting and stopping the scan, directly in the Experiment class:

%! Suppress = Ellipsis
\begin{minted}{python}
    import threading
    [...]

    def start_scan(self):
        self.scan_thread = threading.Thread(target=self.do_scan)
        self.scan_thread.start()

    def stop_scan(self):
        self.keep_running = False
\end{minted}

Now you can update the \textbf{run\_experiment.py} script to make it look much better:

\begin{minted}{python}
experiment.start_scan()
while experiment.is_running:
    print('Experiment Running')
    sleep(1)
experiment.finalize()
\end{minted}

This code is clean and easy to understand. One of the advantages is that it also gives you the freedom to decide whether you want to run the scan on its own thread or not. There's only one more detail, and that is that if you finalize the experiment while the scan is running, then you may face some issues trying to read from a closed device. It's better to update the finalize method:

%! Suppress = Ellipsis
\begin{minted}{python}
    def finalize(self):
        print('Finalizing Experiment')
        self.stop_scan()
        while self.is_running:
            sleep(.1)
        [...]
\end{minted}

You stop the scan, and then you wait until you're sure it has finished. It's important because the delay between data points may be significant, or because acquiring data takes a long time, such as what happens with long exposure times for cameras. Once you know the scan stopped, then you continue to close the communication.

\subsection{Threading and Jupyter notebooks}\label{subsec:jupyter}
Running experiments on Jupyter Notebooks can be a perfect solution to combine the generation of data, its documentation, and its analysis in only one tool. If appropriately used, Jupyter notebooks can be an excellent resource for the experimentalist. However, real-time plotting of data within notebooks is virtually impossible. Building interactive tools on top of a notebook are very complicated, and that is why you chose to follow a different path for the last chapters.

However, for people who are already using Jupyter, it's worth mentioning that how you developed the Experiment and Device models makes them readily available to be incorporated into a notebook. Moreover, the use of threads directly built in the class allows us to run the scan in one cell and simultaneously plot the data on another cell. It can help prototype programs very quickly and can also be the starting point for plugging an experiment directly to the data analysis pipeline.

\section{Conclusions}\label{sec:conclusions-run-experiment}
This chapter aims at polishing some of the details that you were missing to be able to perform a scan in a robust way. The most exciting aspect of the chapter is the inclusion of threads to be able to run a scan and still maintain control of the program. You quickly saw how to plot while the scan runs, and took a look at the problems that can appear when you run multiple threads. You explored some new strategies to prevent a second scan from starting, and for stopping the scan without the risk of losing data.

Threads open a lot of different possibilities for Python developers, not only for lab applications. They are, however, a complex topic that you need to take seriously. The pattern you decided to follow in this chapter, exchanging information between main and child threads using attributes of a class is straightforward, but also prone to problems. You believe that in the context of controlling an experiment, there's rarely the need to go beyond what you've done. It does not mean that you shouldn't keep an eye in case problems arise.
